%%%%%%%%%%%%%%%%%%%%%%%%%%%%%%%%%%%%%%%%%%%%%%%%%%%%%%%
%% Bachelor's & Master's Thesis Template             %%
%% Copyleft by Artur M. Brodzki & Piotr Woźniak      %%
%% Faculty of Electronics and Information Technology %%
%% Warsaw University of Technology, 2019-2020        %%
%%%%%%%%%%%%%%%%%%%%%%%%%%%%%%%%%%%%%%%%%%%%%%%%%%%%%%%

\documentclass[
    left=2.5cm,         % Sadly, generic margin parameter
    right=2.5cm,        % doesnt't work, as it is
    top=2.5cm,          % superseded by more specific
    bottom=3cm,         % left...bottom parameters.
    bindingoffset=6mm,  % Optional binding offset.
    nohyphenation=false % You may turn off hyphenation, if don't like.
]{eiti/eiti-thesis}

\langeng % Dla języka polskiego mamy \langpol
\graphicspath{{img/}}             % Katalog z obrazkami.
\addbibresource{bibliografia.bib} % Plik .bib z bibliografią

\begin{document}

%--------------------------------------
% Strona tytułowa
%--------------------------------------
\MasterThesis % Dla pracy inżynierskiej mamy \EngineerThesis
\instytut{Computer Science}
\kierunek{Computer Science}
\specjalnosc{Information System Engineering}
\title{
    Trwały nośnik jako serwis do zarządzania dokumentami
}
\engtitle{ % Tytuł po angielsku do angielskiego streszczenia
    Durable medium as a service for a document automation system
}
\author{Daniel Bigos}
\album{277277}
\promotor{dr inż. Jacek Wytrębowicz}
\date{\the\year}
\maketitle

%--------------------------------------
% Streszczenie po polsku
%--------------------------------------
\cleardoublepage % Zaczynamy od nieparzystej strony
\streszczenie 
Polskie streszczenie
\slowakluczowe Trwały nośnik, Blockchain, Dokument, Podpis elektroniczny

%--------------------------------------
% Streszczenie po angielsku
%--------------------------------------
\newpage
\abstract
English abstract.
\keywords Durable medium, Blockchain, Document, Digital Signature

%--------------------------------------
% Oświadczenie o autorstwie
%--------------------------------------
\cleardoublepage  % Zaczynamy od nieparzystej strony
\pagestyle{plain}
\makeauthorship

%--------------------------------------
% Spis treści
%--------------------------------------
\cleardoublepage % Zaczynamy od nieparzystej strony
\tableofcontents

%--------------------------------------
% Rozdziały
%--------------------------------------
\cleardoublepage % Zaczynamy od nieparzystej strony
\pagestyle{headings}

\newpage
\section{Introduction}
\subsection{Motivation}
\subsection{Durable medium}
\subsection{Blockchain}
\subsection{Public Key Infrastructure}
\subsection{eIDAS}
\newpage
\section{Summary}

%--------------------------------------------
% Literatura
%--------------------------------------------
\cleardoublepage % Zaczynamy od nieparzystej strony
\printbibliography

%--------------------------------------------
% Spisy (opcjonalne)
%--------------------------------------------
\newpage
\pagestyle{plain}

%--------------------------------------------
% Akronimy (opcjonalne)
%--------------------------------------------
% //AB
\vspace{0.8cm}
\acronymlist
% \acronym{PW}{Politechnika Warszawska}
% \acronym{WEIRD}{ang. \emph{Western, Educated, Industrialized, Rich and Democratic}}

%--------------------------------------------
% Symbole (opcjonalne)
%--------------------------------------------
\listoffigurestoc     % Spis rysunków.
\vspace{1cm}          % vertical space


%--------------------------------------------
% Tabele (opcjonalne)
%--------------------------------------------
\listoftablestoc      % Spis tabel.
\vspace{1cm}          % vertical space


\end{document}
